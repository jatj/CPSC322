\documentclass{article}

\usepackage{fullpage}
\usepackage{color}
\usepackage{amsmath}
\usepackage{url}
\usepackage{verbatim}
\usepackage{graphicx}
\usepackage{parskip}
\usepackage{amssymb}
\usepackage{nicefrac}
\usepackage{algorithm2e} % pseudo-code
\usepackage{booktabs}
\usepackage{siunitx}
\usepackage{caption}
\usepackage{subcaption}
\usepackage{bookmark}
\usepackage{hyperref}
\usepackage{enumitem}
\usepackage[procnames]{listings}

\def\rubric#1{\gre{Rubric: \{#1\}}}{}

% Colors
\definecolor{blu}{rgb}{0,0,1}
\def\blu#1{{\color{blu}#1}}
\definecolor{gre}{rgb}{0,.5,0}
\def\gre#1{{\color{gre}#1}}
\definecolor{red}{rgb}{1,0,0}
\def\red#1{{\color{red}#1}}
\definecolor{ans}{rgb}{0,.5,0}%{0.545,0.27,0.074}
\def\ans#1{{\color{ans}#1}}
\definecolor{keywords}{RGB}{255,0,90}
\definecolor{comments}{RGB}{0,0,113}
\definecolor{red}{RGB}{160,0,0}
\definecolor{green}{RGB}{0,150,0}
\def\norm#1{\|#1\|}

% Configuration
\usepackage[font={color=ans,bf},figurename=Fig.,labelfont={it}]{caption}
\lstset{
    language=Python, 
    basicstyle=\ttfamily\small, 
    keywordstyle=\color{keywords},
    commentstyle=\color{comments},
    stringstyle=\color{red},
    showstringspaces=false,
    identifierstyle=\color{green},
    procnamekeys={def,class}
}

% Math
\def\R{\mathbb{R}}
\def\argmax{\mathop{\rm arg\,max}}
\def\argmin{\mathop{\rm arg\,min}}
\newcommand{\mat}[1]{\begin{bmatrix}#1\end{bmatrix}}
\newcommand{\alignStar}[1]{\begin{align*}#1\end{align*}}
\def\half{\frac 1 2}
\def\cond{\; | \;}

% LaTeX
\newcommand{\fig}[2]{\includegraphics[width=#1\textwidth]{#2}}
\newcommand{\centerfig}[2]{\begin{center}\includegraphics[width=#1\textwidth]{#2}\end{center}}
\newcommand{\matCode}[1]{\lstinputlisting[language=Matlab]{a2f/#1.m}}
\def\items#1{\begin{itemize}#1\end{itemize}}
\def\enum#1{\begin{enumerate}#1\end{enumerate}}


\begin{document}

\title{CPSC 322 Assignment 1}
\author{
    José Abraham Torres Juárez - 79507828 \\
    Gustavo Martin - 62580121
}
\date{}
\maketitle
\vspace{-2em}

% ################################## QUESTION 1 ##################################

\section{Question 1 (27points): Comparing Search Algorithms}
\centerfig{1}{../figs/1_graph}

\subsection{Depth-first search}

\subsection{Breadth-first search}

\subsection{A$^*$}

\subsection{Branch-and-bound}

\subsection{}

\begin{enumerate}[label=(\alph*)]
    \item Did BFS and B\&Bfind the optimal solutionfor this graph?
    \item Are BFSand B\&B optimal in general?Explain your answer. 
    \item Did B\&B expandfewer nodes thanA*? Explain if your answer is true in general for these two algorithms and why.
\end{enumerate}

% ################################## QUESTION 2 ##################################

\clearpage
\section{Question 2 (36points): Uninformed Search: Peg Solitaire}
\centerfig{1}{../figs/2_solitaire}
\centerfig{0.8}{../figs/3_solitaire}

\subsection{Represent peg solitaire as a search problem}
\begin{enumerate}[label=(\alph*)]
    \item How would you represent a node/state?
    \item In your representation, what is the goal node?
    \item How would you representthe arcs?
    \item How many possibleboard states are there?Note: this is notthe same as the number of “valid” or “reachable” game states, which is a much more challenging problem.
\end{enumerate}

\subsection{The search tree:}
\begin{enumerate}[label=(\alph*)]
    \item Write out the first three levels (counting the root as level 1) of the search tree. (Only label the arcs; labeling the nodes would be too much work).
    \item What can you say about the length of the solution(s)?
\end{enumerate}

\subsection{The search algorithm:}
\begin{enumerate}[label=(\alph*)]
    \item What kind of search algorithm would you use for this problem? Justify your answer
    \item Would you use cycle-checking? Justify your answer.
    \item Would you use multiple-path-pruning? Justify your answer.
\end{enumerate}

% ################################## QUESTION 3 ##################################

\clearpage
\section{Question 3 (24 Points) Free Cell }

\subsection{Represent this as a search problem}
\begin{enumerate}[label=(\alph*)]
    \item How would you represent a node/state?
    \item In your representation, what is the goal node?
    \item How would you representthe arcs?
\end{enumerate}

\subsection{}
Give  an  admissible  heuristicfor  this  problem;  explain  why  your  heuristic  is  admissible. More points will be given for tighter lower bounds; for example, h=0 is a trivial (and useless) heuristic, and thus it is not acceptable.

% ################################## QUESTION 4 ##################################

\clearpage
\section{Question 4 (15points) Modified Heuristics}

\subsection{Reduce h(n), trying in three different ways:}

\subsection{Set h(n) as the exact distance from n to a goal}
To do this, you should add the costs of arcs on the optimal path from every node to the goal node, and set the sum as the heuristic function of the node. 
\begin{enumerate}[label=(\alph*)]
    \item Can A* still find the optimal path? 
    \item Is the efficiency of A* increased or reduced? 
    \item Try to draw a more general conclusion regarding the changes in efficiency and optimality. 
\end{enumerate}

\subsection{Increase h(n), also trying in three different ways}
\begin{enumerate}[label=(\alph*)]
    \item When can A* still find the optimal path?
    \item When is the efficiency of A* improved or reduced? 
    \item Try to draw a more general conclusion regarding the changes in efficiency and optimality.
\end{enumerate}

\end{document}
