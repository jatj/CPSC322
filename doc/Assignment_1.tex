\documentclass{article}

\usepackage{fullpage}
\usepackage{color}
\usepackage{amsmath}
\usepackage{url}
\usepackage{verbatim}
\usepackage{graphicx}
\usepackage{parskip}
\usepackage{amssymb}
\usepackage{nicefrac}
\usepackage{algorithm2e} % pseudo-code
\usepackage{booktabs}
\usepackage{siunitx}
\usepackage{caption}
\usepackage{subcaption}
\usepackage{bookmark}
\usepackage{hyperref}
\usepackage{enumitem}
\usepackage[procnames]{listings}

\def\rubric#1{\gre{Rubric: \{#1\}}}{}

% Colors
\definecolor{blu}{rgb}{0,0,1}
\def\blu#1{{\color{blu}#1}}
\definecolor{gre}{rgb}{0,.5,0}
\def\gre#1{{\color{gre}#1}}
\definecolor{red}{rgb}{1,0,0}
\def\red#1{{\color{red}#1}}
\definecolor{ans}{rgb}{0,.5,0}%{0.545,0.27,0.074}
\def\ans#1{{\color{ans}#1}}
\definecolor{keywords}{RGB}{255,0,90}
\definecolor{comments}{RGB}{0,0,113}
\definecolor{red}{RGB}{160,0,0}
\definecolor{green}{RGB}{0,150,0}
\def\norm#1{\|#1\|}

% Configuration
\usepackage[font={color=ans,bf},figurename=Fig.,labelfont={it}]{caption}
\lstset{
    language=Python, 
    basicstyle=\ttfamily\small, 
    keywordstyle=\color{keywords},
    commentstyle=\color{comments},
    stringstyle=\color{red},
    showstringspaces=false,
    identifierstyle=\color{green},
    procnamekeys={def,class}
}

% Math
\def\R{\mathbb{R}}
\def\argmax{\mathop{\rm arg\,max}}
\def\argmin{\mathop{\rm arg\,min}}
\newcommand{\mat}[1]{\begin{bmatrix}#1\end{bmatrix}}
\newcommand{\alignStar}[1]{\begin{align*}#1\end{align*}}
\def\half{\frac 1 2}
\def\cond{\; | \;}

% LaTeX
\newcommand{\fig}[2]{\includegraphics[width=#1\textwidth]{#2}}
\newcommand{\centerfig}[2]{\begin{center}\includegraphics[width=#1\textwidth]{#2}\end{center}}
\newcommand{\matCode}[1]{\lstinputlisting[language=Matlab]{a2f/#1.m}}
\def\items#1{\begin{itemize}#1\end{itemize}}
\def\enum#1{\begin{enumerate}#1\end{enumerate}}


\begin{document}

\title{CPSC 322 Assignment 1}
\author{
    José Abraham Torres Juárez - 79507828 \\
    Gustavo Martin - 62580121
}
\date{}
\maketitle
\vspace{-2em}

% ################################## QUESTION 1 ##################################

\section{Question 1 (27points): Comparing Search Algorithms}
% \centerfig{1}{../figs/1_graph}

\subsection{Depth-first search}
\begin{enumerate}[label=(\alph*)]
    \item What nodes are expanded by the algorithm?
    \ans{
        \[\{a, b, c, d, e, f, g, h, i\}\]
    }
    \item What path is returned by the algorithm?
    \ans{
        \[a->b->c->d->e->f->g->h->i->z\]
    }
    \item What is the cost of this path?
    \ans{
        64
    }
\end{enumerate}

\subsection{Breadth-first search}
\begin{enumerate}[label=(\alph*)]
    \item What nodes are expanded by the algorithm?
    \ans{
        \[\{a, b, e, c, f, g, d, f, g, h, i\}\]
    }
    \item What path is returned by the algorithm?
    \ans{
        \[a->e->g->z\]
    }
    \item What is the cost of this path?
    \ans{
        16
    }
\end{enumerate}

\subsection{A$^*$}
\begin{enumerate}[label=(\alph*)]
    \item What nodes are expanded by the algorithm?
    \ans{
        \[\{a, e, b, g, c, f, d\}\]
    }
    \item What path is returned by the algorithm?
    \ans{
        \[a->e->g->z\]
    }
    \item What is the cost of this path?
    \ans{
        16
    }
\end{enumerate}

\subsection{Branch-and-bound}

\subsection{}

\begin{enumerate}[label=(\alph*)]
    \item Did BFS and B\&Bfind the optimal solutionfor this graph?
    \item Are BFSand B\&B optimal in general?Explain your answer. 
    No, BFS is not optimal if the costs of the arcs are different. And B\&B only if 
    the heuristcs are admissible.
    \item Did B\&B expandfewer nodes than A*? Explain if your answer is true in general for these two algorithms and why.
\end{enumerate}

% ################################## QUESTION 2 ##################################

\clearpage
\section{Question 2 (36points): Uninformed Search: Peg Solitaire}
% \centerfig{1}{../figs/2_solitaire}
% \centerfig{0.8}{../figs/3_solitaire}

\subsection{Represent peg solitaire as a search problem}
\begin{enumerate}[label=(\alph*)]
    \item How would you represent a node/state?
    \ans{
        As a matrix representing the the positions of all the pegs in the board. A peg would 
        be represented as a $1$, a free hole would be represented as a $0$ and a place 
        where the peg can not move would be represented by a $-1$.\\ \\ The following matrix 
        is the representation of the initial state of the game.
        \[ \left[ 
        \begin{array}{ccccccc}
        -1 & -1 & 1 & 1 & 1 & -1 & -1 \\
        -1 & -1 & 1 & 1 & 1 & -1 & -1 \\
        1 & 1 & 1 & 1 & 1 & 1 & 1 \\
        1 & 1 & 1 & 0 & 1 & 1 & 1 \\
        1 & 1 & 1 & 1 & 1 & 1 & 1 \\
        -1 & -1 & 1 & 1 & 1 & -1 & -1 \\
        -1 & -1 & 1 & 1 & 1 & -1 & -1 
        \end{array} 
        \right]\] 
    }
    \item In your representation, what is the goal node?
    \ans{
        The goal node would have the following represendation.
        \[ \left[ 
        \begin{array}{ccccccc}
        -1 & -1 & 0 & 0 & 0 & -1 & -1 \\
        -1 & -1 & 0 & 0 & 0 & -1 & -1 \\
        0 & 0 & 0 & 0 & 0 & 0 & 0 \\
        0 & 0 & 0 & 1 & 0 & 0 & 0 \\
        0 & 0 & 0 & 0 & 0 & 0 & 0 \\
        -1 & -1 & 0 & 0 & 0 & -1 & -1 \\
        -1 & -1 & 0 & 0 & 0 & -1 & -1 
        \end{array} 
        \right]\] 
    }
    \item How would you representthe arcs?
    \ans{
        We would use the following notation $ij.i'j'$, being $i$ the row, $j$ the column of 
        the peg that is moving to the column $i'$ and row $j'$, e.g., an arc with the label 
        31.33 will represent that a peg in row 3 column 1 will be moved to row 3 column 3.
    }
    \item How many possible board states are there?Note: this is notthe same as the number of “valid” or “reachable” game states, which is a much more challenging problem.
    \ans{
        There are 33 holes in the board, and the domain of this holes is binary {1,0}, 
        either there is a peg on the hole or not. So the number of possible states would 
        be $2^{33}$.
    }
\end{enumerate}

\subsection{The search tree:}
\begin{enumerate}[label=(\alph*)]
    \item Write out the first three levels (counting the root as level 1) of the search tree. (Only label the arcs; labeling the nodes would be too much work).
    \ans{
        The following diagram is the representation of how the search tree would look.
        \centerfig{0.8}{../figs/q2_2_a}
    }
    \item What can you say about the length of the solution(s)?
    \ans{
        The solution must be at level 32, because at every step from node to node, we 
        lose one peg, we start with 32 at the goal is to get to 1 peg in all the board.
    }

\end{enumerate}

\subsection{The search algorithm:}
\begin{enumerate}[label=(\alph*)]
    \item What kind of search algorithm would you use for this problem? Justify your answer
    \ans{
        An uninformed algorithm, because we neither have difference in the costs of the paths
        nor heuristcs.
    }
    \item Would you use cycle-checking? Justify your answer.
    \ans{
        No, you can't undo a movement so there is no posibility for cycles.
    }
    \item Would you use multiple-path-pruning? Justify your answer.
    \ans{
        No, because it is not possible to arrive at the same state with less movements. 
        We can prove this because at every level the number of pegs is reduced by 1, at 
        level 1 we have 32 pegs, at level 2 31 pegs, at 3 we have 30 pegs and so on and 
        so forth. So it is guaranted that they will not be a shorter path to get to any 
        state.
    }
\end{enumerate}

% ################################## QUESTION 3 ##################################

\clearpage
\section{Question 3 (24 Points) Free Cell }

\subsection{Represent this as a search problem}
\begin{enumerate}[label=(\alph*)]
    \item How would you represent a node/state?
    \item In your representation, what is the goal node?
    \item How would you representthe arcs?
\end{enumerate}

\subsection{}
Give  an  admissible  heuristicfor  this  problem;  explain  why  your  heuristic  is  admissible. More points will be given for tighter lower bounds; for example, h=0 is a trivial (and useless) heuristic, and thus it is not acceptable.

% ################################## QUESTION 4 ##################################

\clearpage
\section{Question 4 (15points) Modified Heuristics}

\subsection{Reduce h(n), trying in three different ways:}

\subsection{Set h(n) as the exact distance from n to a goal}
To do this, you should add the costs of arcs on the optimal path from every node to the goal node, and set the sum as the heuristic function of the node. 
\begin{enumerate}[label=(\alph*)]
    \item Can A* still find the optimal path? 
    \item Is the efficiency of A* increased or reduced? 
    \item Try to draw a more general conclusion regarding the changes in efficiency and optimality. 
\end{enumerate}

\subsection{Increase h(n), also trying in three different ways}
\begin{enumerate}[label=(\alph*)]
    \item When can A* still find the optimal path?
    \item When is the efficiency of A* improved or reduced? 
    \item Try to draw a more general conclusion regarding the changes in efficiency and optimality.
\end{enumerate}

\end{document}
