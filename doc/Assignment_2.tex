\documentclass{article}

\usepackage{fullpage}
\usepackage{color}
\usepackage{amsmath}
\usepackage{url}
\usepackage{verbatim}
\usepackage{graphicx}
\usepackage{parskip}
\usepackage{amssymb}
\usepackage{nicefrac}
\usepackage{algorithm2e} % pseudo-code
\usepackage{booktabs}
\usepackage{siunitx}
\usepackage{caption}
\usepackage{subcaption}
\usepackage{bookmark}
\usepackage{hyperref}
\usepackage{enumitem}
\usepackage[procnames]{listings}

\def\rubric#1{\gre{Rubric: \{#1\}}}{}

% Colors
\definecolor{blu}{rgb}{0,0,1}
\def\blu#1{{\color{blu}#1}}
\definecolor{gre}{rgb}{0,.5,0}
\def\gre#1{{\color{gre}#1}}
\definecolor{red}{rgb}{1,0,0}
\def\red#1{{\color{red}#1}}
\definecolor{ans}{rgb}{0,.5,0}%{0.545,0.27,0.074}
\def\ans#1{{\color{ans}#1}}
\definecolor{keywords}{RGB}{255,0,90}
\definecolor{comments}{RGB}{0,0,113}
\definecolor{red}{RGB}{160,0,0}
\definecolor{green}{RGB}{0,150,0}
\def\norm#1{\|#1\|}

% Configuration
\usepackage[font={color=ans,bf},figurename=Fig.,labelfont={it}]{caption}
\lstset{
    language=Python, 
    basicstyle=\ttfamily\small, 
    keywordstyle=\color{keywords},
    commentstyle=\color{comments},
    stringstyle=\color{red},
    showstringspaces=false,
    identifierstyle=\color{green},
    procnamekeys={def,class}
}

% Math
\def\R{\mathbb{R}}
\def\argmax{\mathop{\rm arg\,max}}
\def\argmin{\mathop{\rm arg\,min}}
\newcommand{\mat}[1]{\begin{bmatrix}#1\end{bmatrix}}
\newcommand{\alignStar}[1]{\begin{align*}#1\end{align*}}
\def\half{\frac 1 2}
\def\cond{\; | \;}

% LaTeX
\newcommand{\fig}[2]{\includegraphics[width=#1\textwidth]{#2}}
\newcommand{\centerfig}[2]{\begin{center}\includegraphics[width=#1\textwidth]{#2}\end{center}}
\newcommand{\matCode}[1]{\lstinputlisting[language=Matlab]{a2f/#1.m}}
\def\items#1{\begin{itemize}#1\end{itemize}}
\def\enum#1{\begin{enumerate}#1\end{enumerate}}


\begin{document}

\title{CPSC 322 Assignment 1}
\author{
    Jose Abraham Torres Juarez - 79507828 \\
    Gustavo Martin - 62580121
}
\date{}
\maketitle
\vspace{-2em}

% ################################## QUESTION 1 ##################################

\section{Question 1 (21points): Allocating Developments Problem}

\begin{enumerate}[label=(\alph*)]
    \item Represent this problem as a CSP
    \begin{enumerate}\color{ans}
        \item \textbf{Variables: }
        \[
            \{ hc, bh, ra, gd \}
        \]
        \item \textbf{Domains: }
        \[
            \{ (0,0), (0,1), (0,2), (1,0), (1,1), (1,2), (2,0), (2,1), (2,2) \}
        \]
        \item \textbf{Constraints: }
        \[
            \{ hc \neq ra, hc \neq dg, hc \neq bh, bh \neq ra, bh \neq gd, ra \neq gd,
        \]
        \[
            hc \neq (0,0), bh \neq (0,0), ra \neq (0,0), gd \neq (0,0),
        \]
        \[
            hc \neq (1,2), bh \neq (1,2), ra \neq (1,2), gd \neq (1,2),
        \]
        \[
            !Close(hc,(0,0)), !Close(bh,(0,0)), !Close(hc,gd), !Close(bh,gd)
        \]
        \[
            Close(ra,(1,2)), Close(hc,ra), Close(bh,ra) \}
        \]
        \ans{
            For our constraint definitions we use the following functions:
        }
        \begin{tabbing}
        xxxx\=xxxx\=xxxx\=xxxx\=xxxx\= \kill \color{ans}
        {\bf Function} {\em Close} ($A,B$)\\ [0pt \color{ans}]
        \> // Check if $|| A - B ||_1$ == 1, the Manhattan distance must be 1\\
        \> return $( |A[0] - B[0]| + |A[1] - B[1]| ) == 1$ \\
        \end{tabbing}
    \end{enumerate}
    \item Draw a constraing graph for this problem
    \centerfig{0.8}{../figs/q1.png}
\end{enumerate}

% ################################## QUESTION 2 ##################################
\clearpage
\section{CSP - Search}
\begin{enumerate}[label=(\alph*)]
    \item Show how search can be used to solve this problem, using the variable ordering A, B, C, D, E, F, G, H. \\
    \ans{
        See \hyperref[appendix:CSP Search]{\textbf{Appendix A}} for the implementation of \texttt{CSP}.\\
    }
    \item Is it  possible  to  generate a  smaller  tree? 
    \ans{
    }
    \item Explain why you expect the heuristic in part (b) to be good
    \ans{
    }
\end{enumerate}

% ################################## QUESTION 3 ##################################
\clearpage
\section{CSP - Arc Consistency}
\begin{enumerate}[label=(\alph*)]
    \item Show how arc consistency can be used to solve the scheduling problem in Question 2
    \ans{
    }
    \item Use domain splitting to solve this problem
    \ans{
    }
    \item Constraint satisfaction problems can become extremely large and complex.
    \ans{
    }
\end{enumerate}

% ################################## QUESTION 4 ##################################
\clearpage
\section{CSP - Stochastic Local Search}
\begin{enumerate}[label=(\alph*)]
    \item For one particular run, make SLS select any variable that is involved in an unsatisfied constraint, and select a value that results in the minimum number of unsatisfied constraints. Report the initialized values.  At each step, report which variable is changed, its new value, and the resulting number of unsatisfied arcs. (You only need to do this for 5 steps).
    \ans{
    }
    \item Using batch runs, compare and explain the result of the following settings:
    \begin{enumerate}[label=(\roman*)]
        \item Select a variable involved in the maximum number of unsatisfied constraints, and the best value.
        \ans{
        }
        \item Select any variable that is involved in unsatisfied constraints, and the best value.
        \ans{
        }
        \item Select a variable at random, and the best value.
        \ans{
        }
        \item A probabilistic mix of i. and ii.Try a few probabilities and report on the best one you find
        \ans{
        }
    \end{enumerate}
    \item Pick one of the methods(ii)-(iv)from part (b) and run it repeatedly (5 runs should be enough, but you may do more runs if you wish). State which method you chose. Give a screenshot of the results; then think about, describe and explain whatyou observe.
    \ans{
    }
    \item How important is it to choose the value that results in the fewest unsatisfied constraints as opposed to choosing a value at random? Justify your answer with evidence.
    \ans{
    }
    \item For the best variable/best value method from part(b), allow random resets (for example, after 50 steps). Give a screenshot comparing the results to what happens if random resets are not allowed, and explain how this affects the performance of the algorithm, and why.
    \ans{
    }
\end{enumerate}

\bookmarksetup{startatroot}

\appendix

\section{Appendix: CSP as Search implementation}
\label{appendix:CSP Search}
\begin{center}
    \lstinputlisting{../code/q2.py}
\end{center}

\end{document}