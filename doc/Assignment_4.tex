\documentclass{article}

\usepackage{fullpage}
\usepackage{color}
\usepackage{amsmath}
\usepackage{url}
\usepackage{verbatim}
\usepackage{graphicx}
\usepackage{parskip}
\usepackage{amssymb}
\usepackage{nicefrac}
\usepackage{algorithm2e} % pseudo-code
\usepackage{booktabs}
\usepackage{siunitx}
\usepackage{caption}
\usepackage{subcaption}
\usepackage{bookmark}
\usepackage{hyperref}
\usepackage{enumitem}
\usepackage[procnames]{listings}
\usepackage{amssymb}
\usepackage{multicol}
\usepackage[linguistics]{forest}
\usepackage{soul}
\forestset{
sn edges/.style={
    for tree={
        rectangle, draw,
        s sep+ = 2pt,
        l sep+ = 25pt,
        parent anchor=south,
        edge path={
        \noexpand\path [\forestoption{edge}] (!u.parent anchor) -- ++(0,-15pt) -| (.child anchor)\forestoption{edge label};
        },
        edge={->}
    }
}
}

\def\rubric#1{\gre{Rubric: \{#1\}}}{}
\newcommand{\stripNot}[1]{$\overline{\text{#1}}$}

% Colors
\definecolor{blu}{rgb}{0,0,1}
\def\blu#1{{\color{blu}#1}}
\definecolor{gre}{rgb}{0,.5,0}
\def\gre#1{{\color{gre}#1}}
\definecolor{red}{rgb}{1,0,0}
\def\red#1{{\color{red}#1}}
\definecolor{ans}{rgb}{0,.5,0}%{0.545,0.27,0.074}
\def\ans#1{{\color{ans}#1}}
\definecolor{keywords}{RGB}{255,0,90}
\definecolor{comments}{RGB}{0,0,113}
\definecolor{red}{RGB}{160,0,0}
\definecolor{green}{RGB}{0,150,0}
\def\norm#1{\|#1\|}

% Configuration
\usepackage[font={color=ans,bf},figurename=Fig.,labelfont={it}]{caption}
\lstset{
    language=Python, 
    basicstyle=\ttfamily\small, 
    keywordstyle=\color{keywords},
    commentstyle=\color{comments},
    stringstyle=\color{red},
    showstringspaces=false,
    identifierstyle=\color{green},
    procnamekeys={def,class}
}

% Math
\def\R{\mathbb{R}}
\def\argmax{\mathop{\rm arg\,max}}
\def\argmin{\mathop{\rm arg\,min}}
\newcommand{\mat}[1]{\begin{bmatrix}#1\end{bmatrix}}
\newcommand{\alignStar}[1]{\begin{align*}#1\end{align*}}
\def\half{\frac 1 2}
\def\cond{\; | \;}

% LaTeX
\newcommand{\fig}[2]{\includegraphics[width=#1\textwidth]{#2}}
\newcommand{\centerfig}[2]{\begin{center}\includegraphics[width=#1\textwidth]{#2}\end{center}}
\newcommand{\matCode}[1]{\lstinputlisting[language=Matlab]{a2f/#1.m}}
\def\items#1{\begin{itemize}#1\end{itemize}}
\def\enum#1{\begin{enumerate}#1\end{enumerate}}


\begin{document}

\title{CPSC 322 Assignment 4}
\author{
    Jose Abraham Torres Juarez - 79507828 \\
    Gustavo Martin - 62580121 \\ \\
}
\date{}
\maketitle
\vspace{-2em}

% ################################## QUESTION 1 ##################################

\section{Question 1 [20 points] Probabilities}
\centerfig{0.35}{../figs/fig1.png}
Consider the above joint probability distribution over the three random variables A, B, C. \\
\begin{enumerate}[label=(\alph*)]
    \item \textbf{5 points} You are told that it contains only one incorrect value. What is that value? Do you have sufficient information to fix it? If that is the case, how would you fix it? If not, why not? \\
    \ans{
        If we add the probabilities of all the rows, it must sum to 1, but instead it sums to 1.4. 
        Assuming that the table only contains one incorrect value, the incorrect probability should have 
        a value of 0.4, to make the table valid. \\ \\
        Thus we can fix the table by setting the probability of (A$\bar{\text{B}}$C) to 0
    } 
    \item \textbf{8 points} With the joint resulting from the previous question (after you fixed the error), compute the marginal probability distributions for P(A) and P(B). Are A and B independent? Also, what is the value of P(A=T $|$ B=F)?  \\
    \ans{
        \begin{center}
            P(A) = 0.1 + 0.1 + 0 + 0.3 = 0.5 \\
            P(B) = 0.1 + 0.1 + 0.1 + 0.2 = 0.5 \\
            P(A$|$B) = P(A, B)/P(B) = (0.1+0.1)/0.5 = 0.4 \\
        \end{center}
        From the calculations above, we can notice that P(A) != P(A$|$B), this imply that A and B 
        are not independent. \\
        \begin{center}
            P(A$|$$\bar{\text{B}}$) = P(A and $\bar{\text{B}}$)/P($\bar{\text{B}}$) = (0+0.3)/0.5 = 0.6 \\
        \end{center}
    }
    \item \textbf{7 points} \textbf{Without performing any calculations}, is it true that P(A) * P(B $|$ A) * P(C $|$ A, B) = P(C) * P(B $|$ C) * P(A $|$ B, C)? Why or why not?\\
    \ans{
        \begin{center}
            P(A) = 0.5 \\
            P(C) = 0.1 + 0 + 0.1 + 0.1 = 0.3 \\
            P(B$|$A) = P(B, A)/P(A) = (0.1+0.1)/0.5 = 0.4 \\
            P(C$|$AB) = P(A, B, C)/P(A, B) = 0.1/(0.1+0.1) = 0.5 \\
            P(B$|$C) = P(B, C)/P(C) = (0.1+0.1)/0.3 $\approx$ 0.66667 \\
            P(A$|$BC) = P(A, B, C)/P(B, C) = 0.1/(0.1+0.1) = 0.5 \\
            ------------------------------------------------------------------------ \\
            P(A) * P(B $|$ A) * P(C $|$ A, B) = 0.5 * 0.4 * 0.5 = 0.1 \\
            P(C) * P(B $|$ C) * P(A $|$ B, C) = 0.3 * 0.66667 * 0.5 = 0.1 \\
        \end{center}
        The calculation above prove that the given formula where true, but we can also prove this by 
        simplifying the provided formula.
        \begin{center}
            P(A) * P(B $|$ A) * P(C $|$ A, B) = \st{P(A)} * $\frac{\text{\st{P(A, B)}}}{\text{\st{P(A)}}}$ * $\frac{\text{P(C, A, B)}}{\text{\st{P(A, B)}}}$ = P(C, A, B) \\
            P(C) * P(B $|$ C) * P(A $|$ B, C) = \st{P(C)} * $\frac{\text{\st{P(C, B)}}}{\text{\st{P(C)}}}$ * $\frac{\text{P(A, C, B)}}{\text{\st{P(C, B)}}}$ = P(A, C, B) \\
        \end{center}
        So we can reduce everything to the next clause P(C, A, B) = P(A, B, C), we know that must be true 
        because the intersection of the three variables will always be the same.
    }
\end{enumerate}

% ################################## QUESTION 2 ##################################
\clearpage
\section{Question 2 [20 points] Bayes' rule}
You’ve taken your prized sports car into a repair shop for a routine maintenance check. As part of a standard set of diagnostics, the car’s onboard computer is checked for logged problems. The mechanic tells you that they have good and bad news. The bad news is that the car’s computer has reported that it has detected a failure in the electrical system that will definitely cause an electrical fire in the near future (thus you may treat the failure and the resulting fire as the same “thing”). The probability of the computer reporting this warning given that there is actually a failure in the electrical system is 0.95, as is the probability that the computer will correctly not report anything if the failure isn’t present. The good news is that this type of failure and the consequent electric fire will only occur in one out of every ten thousand cars of this model on the road. \\
\begin{enumerate}[label=(\alph*)]
    \item \textbf{12 points} What are the chances that an electrical fire will occur in your car if don’t opt to repair it? (Show your calculations as well as giving the final result.)\\
    \ans{
        \begin{center}
            P(Failure) = 1/10000 = 0.0001 \\
            P(Detection $|$ Failure) = 0.95 \\
            P(F $|$ D) = P(FD)/P(D) = (0.0001*0.95)/(0.0001*0.95 + 0.9999*0.05) = \fbox{0.001896} \\
        \end{center}
    }
    \item \textbf{4 points} Repeat the calculation of part (a), but for the case where the failure occurs in one out of every hundred cars of this model.    \\
    \ans{
        \begin{center}
            P(Failure) = 1/100 = 0.01 \\
            P(Detection $|$ Failure) = 0.95 \\
            P(F $|$ D) = P(FD)/P(D) = (0.01*0.95)/(0.01*0.95 + 0.99*0.05) = \fbox{0.1610} \\
        \end{center}
    }
    \item \textbf{4 point} Given the logged report of a problem, why is it good news that the issue is so rare in part (a)?\\
    \ans{
        Because the low probabilities of having a failure, decrease the probabilities 
        of getting a real failure. We can say that it is so rare to get a failure that 
        it is more probable that the computer is detecting a false positive, so the car 
        won't need to be repaired.
    }
\end{enumerate}

% ################################## QUESTION 3 ##################################
\clearpage
\section{Question 3 [15 points] Bayesian/Belief networks}
Consider a probabilistic problem that can be represented by the simple Bayesian Network given below. Let L1 and L2 be Boolean variables, and let C1, C2, and N have 51 possible values each. Compute the representational saving (i.e. how many fewer values you need to store) between this belief network and the joint distribution for this problem. Show your work. \\
\centerfig{0.4}{../figs/fig2}
\ans{
    The joint probability table would need to store 2*2*51*51*51 = 530604 values, which 
    can be reduced to 5406 by using a noisy or representations. That is a reduction of 
    525198.
}

% ################################## QUESTION 4 ##################################
\clearpage
\section{Question 4 [40 points] Variable Elimination}
\centerfig{1}{../figs/fig3}
Carry out variable elimination (VE) on this network to compute P(S|Q=F). \\
\begin{enumerate}[label=\arabic*.]
    \item \textbf{10 marks} indicate which nodes can be pruned (justifying each pruning step), and list the factors (with their values) that VE needs to initially create.
    \ans{
        \begin{enumerate}[label=\arabic*.]
            \item M can be pruned because it is an unobserved leaf
            \item X can be pruned because it is an unobserved leaf
            \item Z can be pruned because it is an unobserved leaf
            \item W can be pruned because it is conditionally independent from the query
            \item V can now be pruned because it is an unobserved leaf
        \end{enumerate}
    }
    \item \textbf{5 marks} Assuming that the elimination ordering is alphabetical, show how the factors and the summations should be ordered. \\
    \ans{
        \[ \sum_U f_7(U,S) \sum_R f_4(R,U) \sum_O f_0(O) f_2(O,R) \sum_N f_1(O,N) f_6(N,R)\]
    }
    \item \textbf{15 marks} step through VE, showing what operations are performed, the resulting intermediate factors (with their values) and how P(S|Q=F) is finally computed (hint: see the example we traced in class and the factor operations we covered in the slides).
    \ans{
        \begin{enumerate}[label=\arabic*.]
            \item Compute intermediate result for $N$, $f_8'(O,N,R)$
            \item Eliminate $N$, result in: $\sum_U f_7(U,S) \sum_R f_4(R,U) \sum_O f_0(O) f_2(O,R) f_8(O,R)$
            \[
                f_8(O,R) = \mat{
                    \text{O} & \text{R} & \text{Val} \\
                    \text{T} & \text{T} & 0.87 \\
                    \text{T} & \text{F} & 0.74 \\
                    \text{F} & \text{T} & 0.76 \\
                    \text{F} & \text{F} & 0.66 \\
                }
            \]
            \item Compute intermediate result for $O$, $f_9'(O,R)$
            \item Eliminate $O$, result in: $\sum_U f_7(U,S) \sum_R f_4(R,U) f_9(R)$
            \[
                f_9(R) = \mat{
                    \text{R} & \text{Val} \\
                    \text{T} & 0.4794 \\
                    \text{F} & 0.3062 \\
                }
            \]
            \item Compute intermediate result for $R$, $f_{10}'(R,U)$
            \item Eliminate $R$, result in: $\sum_U f_7(U,S) f_{10}(U)$
            \[
                f_{10}(U) = \mat{
                    \text{U} & \text{Val} \\
                    \text{T} & 0.4194 \\
                    \text{F} & 0.3661 \\
                }
            \]
            \item Compute intermediate result for $U$, $f_{11}'(U,S)$
            \item Eliminate $U$, result in: $f_{11}(S)$
            \[
                f_{11}(S) = \mat{
                    \text{S} & \text{Val} \\
                    \text{T} & 0.4241 \\
                    \text{F} & 0.3614 \\
                }
            \]
            \item Normalize $f_{11}(S)$
            \[
                f_{11}(S) = \mat{
                    \text{S} & \text{Val} \\
                    \text{T} & 0.54 \\
                    \text{F} & 0.46 \\
                }
            \]
        \end{enumerate}
    }
\end{enumerate}

\end{document}
